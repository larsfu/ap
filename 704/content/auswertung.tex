\section{Auswertung}
\label{sec:Auswertung}
\subsection{\texorpdfstring{$\upgamma$}{γ}-Absorption}
Die aufgenommenen Zählraten werden über die Dicke aufgetragen und es wird eine nichtlineare Ausgleichsrechnung an
\begin{equation}
  N(D) = N_0 \symup{e}^{-\mu D}
\end{equation}
durchgeführt. Die Plots finden sich in Abb.~\ref{fig:Cu}~und~\ref{fig:Pb}. Es wird ein Fehler von
\begin{equation}
  \Delta N = \frac{\sqrt{n}}{\Delta t}
\end{equation}
angenommen. Die Zählrate
\begin{equation}
  N_U = \frac{917}{\SI{900}{\second}} \approx \SI{1.02}{\per\second},
\end{equation}
die ohne Strahlungsquelle ermittelt wurde, wird als Nulleffekt abgezogen. Zur besseren Vergleichbarkeit sind die Messwerte in Abb.~\ref{fig:CuLog}~und~\ref{fig:PbLog} nochmal halblogarithmisch dargestellt.
Es ergiben sich die Absorptionskoeffizienten
\begin{align}
  \mu_\text{Cu} &= \SI{49.8+-0.6}{\per\meter},\\
  \mu_\text{Pb} &= \SI{108 +- 2}{\per\meter},
\shortintertext{sowie}
  N(0)_\text{Cu} &= \SI{147+-1}{\per\second},\\
  N(0)_\text{Pb} &= \SI{154+-3}{\per\second}.
\end{align}
Durcheinsetzen von \eqref{eqn:sigma_com} und \eqref{eqn:r_e} in \eqref{eqn:mu_com} und Ausnutzen von
\begin{equation}
  \frac{\rho}{M} = \frac{1}{V_\text{mol}}
\end{equation}
ergibt sich der Zusammenhang
\begin{equation}
  \mu_\text{com} = \frac{z N_A}{V_\text{mol}} \cdot 2 \pi \left(\frac{e^2}{4\pi \epsilon_0 m_e c^2}\right)^2 \left(\frac{1+\epsilon}{\epsilon^2} \left(\frac{2(1+\epsilon)}{1+2\epsilon^2} - \frac{1}{\epsilon} \ln(1+2\epsilon)\right) \frac{1}{2\epsilon} \ln (1+2\epsilon) - \frac{1+3\epsilon}{(1+2\epsilon)^2}\right).
\end{equation}
Mit den Ordnungszahlen
\begin{align}
  z_\text{Cu} &= 29 \\
  z_\text{Pb} &= 82,
\end{align}
den molaren Volumen
\begin{align}
  V_\text{mol}^\text{Cu} &= \SI{7.11e-6}{\cubic\meter\per\mol}\\
  V_\text{mol}^\text{Pb} &= \SI{18.26e-6}{\cubic\meter\per\mol}
\end{align}
aus \cite{chemie.de} und den Naturkonstanten aus \cite{codata} ergibt sich ein Compton-Absorptionskoeffizient von
\begin{align}
  \mu_\text{com}^\text{Cu} = \SI{62.93}{\per\meter} \\
  \mu_\text{com}^\text{Pb} = \SI{69.28}{\per\meter}.
\end{align}
Die Messdaten sind in den Tabellen \ref{tab:datenCu}~und~\ref{tab:datenPb} zu finden.
\fig{build/Cu}{Gemessene Zählrate in Abhängigkeit von der Dicke der Kupfer-Abschirmung. Dazu nichtlinearer Fit.}{Cu}
\fig{build/Cu_log}{Logarithmische Darstellung von Abb.~\ref{fig:Cu}}{CuLog}
\fig{build/Pb}{Gemessene Zählrate in Abhängigkeit von der Dicke der Blei-Abschirmung. Dazu nichtlinearer Fit.}{Pb}
\fig{build/Pb_log}{Logarithmische Darstellung von Abb.~\ref{fig:Pb}}{PbLog}

\input{build/Cu.tex}
\input{build/Pb.tex}

\FloatBarrier
\subsection{\texorpdfstring{$\upbeta$}{β}-Absorption}
Da die die Mechanismen bei der $\upbeta$-Strahlung ungleich komplexer sind, kann aus der Absorptionskurve lediglich die maximale Eindringtiefe und damit die maximale Energie der Strahlung bestimmt werden. Dazu legt man an die zwei Bereiche der Absorptionskurve jeweils mittels linearer Regression eine Gerade an und bestimmt den Schnittpunkt der beiden Geraden, siehe Abb.~\ref{fig:beta}. Die Zählrate
\begin{equation}
  N_U = \frac{324}{\SI{900}{\second}} \approx \SI{0.36}{\per\second},
\end{equation}
die ohne Strahlungsquelle ermittelt wurde, wird als Nulleffekt abgezogen. Es ergibt sich der Zusammenhang
\begin{equation}
  D_\text{max} = \frac{b_2 -b_1}{a_1-a_2}
\end{equation}
mit den Geradenparametern $a_i$ und $b_i$. Daraus folgt
\begin{align}
  D_\text{max} &= \SI{165+-17}{\micro\meter}
  \shortintertext{und}
  R_\text{max} = \rho \cdot D_\text{max}  &= \SI{0.45+-0.05}{\kilogram\per\meter\squared},
\end{align}
wobei die Dichte $\rho = \SI{2700}{\kilogram\per\cubic\meter}$ aus \cite{chemie.de} stammt.
Nach \eqref{eqn:E_max} folgt daraus eine maximale Energie von
\begin{equation}
  E_\text{max} = \SI{0.61+-0.09}{\mega\electronvolt}.
\end{equation}
Die Messdaten aus diesem Aufgabenteil finden sich in Tabelle~\ref{tab:datenAl}.
\fig{build/beta.pdf}{Beta-Absorptionskurve mit zwei linearen Ausgleichsgeraden.}{beta}
\input{build/Al.tex}
