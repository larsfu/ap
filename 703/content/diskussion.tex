\section{Diskussion}
\label{sec:Diskussion}

Bei der Auftragung der Zählrate $N$ gegen die Spannung $U$ zeigt sich der typische Verlauf einer Charakteristik eines Zählrohrs. Das Plateau ((410-650)\si{\volt}) wird durch einen linearen Zusammenhang  mit einer Steigung von $m=(1,2 \pm 0,2)\%$ beschrieben.  Die Ausgleichsgerade liegt im Bereich der Fehlerbalken der Messwerte, welche in der Ausgleichsrechnung berücksichtig werden. Nur wenige Werte weichen so stark ab, dass sie von der Ausgleichsrechnung ausgeschlossen werden.
Der zeitliche Abstand zwischen Primär- und Nachentladungsimpuls kann beim Ablesen auf dem Oszillographen nur grob abgeschätzt werden, da dies auf dem Oszillographen nur kurze Zeit zu erkennen ist. Er beträgt $\delta t \approx 225 \mu \si{\second}$.
Der durch Ablesen geschätzte Wert für die Totzeit beträgt $ T\approx (150)\mu\si{\second}$, der mittels der Zwei-Quellen-Methode berechnete Wert $T \approx (86 \pm 7)\mu\si{\second}$. Somit liegen die beiden Werte in der gleichen Größenordnung. Der berechnete Wert weicht um $(74,70 \pm 1,3)\%$ vom Abgeschätzten ab.
Die freigesetzte Ladungsmenge $\Delta Q$ hängt linear von der Spannung $U$ ab. Die Ausgleichsgerade hat die Form $\Delta Q = ((3,21 \pm 0.06)\cdot 10^8)\frac{1}{\si{\volt}}x + (893,55 \pm 30,77)\cdot 10^8$.
