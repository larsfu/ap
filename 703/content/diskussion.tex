\section{Diskussion}
\label{sec:Diskussion}
\subsection{Zählrohr-Charakteristik}
Bei der Auftragung der Zählrate $N$ gegen die Spannung $U$ zeigt sich der typische Verlauf einer Charakteristik eines Zählrohrs. Das Plateau ((410-650)\si{\volt}) wird durch einen linearen Zusammenhang  mit einer Steigung von $m=(0,012 \pm 0,002)\%$ beschrieben.  Die Ausgleichsgerade liegt im Bereich der Fehlerbalken der Messwerte, welche in der Ausgleichsrechnung berücksichtig werden. Nur wenige Werte weichen so stark ab, dass sie von der Ausgleichsrechnung ausgeschlossen werden.
\subsection{Primär- und Nachentladungsimpulse}
Der zeitliche Abstand zwischen Primär- und Nachentladungsimpuls kann beim Ablesen auf dem Oszillographen nur grob abgeschätzt werden, da dies auf dem Oszillographen nur kurze Zeit zu erkennen ist.
\subsection{Bestimmung der Totzeit}
Der durch Ablesen geschätzte Wert für die Totzeit beträgt $ T\approx (150)\mu\si{\second}$, der mittels der Zwei-Quellen-Methode berechnete Wert $T \approx (86 \pm 7)\mu\si{\second}$. Somit liegen die beiden Werte in der gleichen Größenordnung. Der berechnete Wert weicht um $(74,70 \pm 1,3)\%$ vom Abgeschätzten ab.
\subsection{Freigesetzte Ladungsmenge}
Die freigesetzte Ladungsmenge $\Delta Q$ hängt linear von der Spannung $U$ ab. Die Ausgleichsgerade hat die Form $\Delta Q = (2,0038 \pm 3,7011)x - (0,0056\pm0,0002)$.
