\section{Auswertung}
\label{sec:auswertung}

\subsection{Güteziffer}
  \label{subsec:güteziffer}
  Es soll die reale Güteziffer der Wärmepumpe ermittelt werden. Diese ergibt sich nach (\ref{eqn:güteziffer1}) und (\ref{eqn:güteziffer2}) zu
  \begin{equation}
    ν = \frac{(m_{1} c_w + m_k c_k)}{N}\frac {\mathrm{d}T_{1}}{\mathrm{d}t}.
  \end{equation}
  Der dort auftretende Differentialquotient $\frac {\mathrm{d}T_{1}}{\mathrm{d}t}$ lässt sich durch einen Fit der Messdaten an eine Funktion der Form
  \begin{equation}
    T(t) = \frac{A t^\alpha}{1+Bt^\alpha} + C
  \end{equation}
  errechnen. Die Messdaten ergeben folgende Koeffizienten (von Einheiten wurde mangels physikalischen Sinns abgesehen):

  \begin{table}[!h]
    \centering
    \caption{Koeffizienten des nichtlinearen Fits.}
    \label{tab:koeffizienten}
    \sisetup{
      round-mode=places,
      round-precision=1,
    }
    \begin{tabular}{
        l@{}
        S[table-format=1.1e2] @{${}\pm{}$} S[table-format=1.1e2]
        S[table-format=1.1e2] @{${}\pm{}$} S[table-format=1.1e2]
        S[table-format=1.1e1] @{${}\pm{}$} S[table-format=1.1e2]
        S[table-format=1.1e0] @{${}\pm{}$} S[table-format=1.1e2]}
      \toprule
      & \multicolumn{2}{c}{$A$}
      & \multicolumn{2}{c}{$B$}
      & \multicolumn{2}{c}{$C$}
      & \multicolumn{2}{c}{$\alpha$} \\
      \midrule
      \input{coefficients.tex}
      \bottomrule
    \end{tabular}
  \end{table}
  Mithilfe der zeitlichen Ableitung von $T(t)$

  \begin{equation}
    \frac{\mathrm{d}T(t)}{\mathrm{d}t} = \frac{αAt^{α - 1} \cdot (1 + B t^α) - A t^α \cdot αBt^{α - 1}}{(1 + B t^ α)^ 2}
  \end{equation}

  und der Wärmekapazität $c_w = \SI{9999(9)}{\joule\per\kilogram\per\kelvin}$ sowie Dichte (unter Normalbedingungen) von Wasser $\rho = \SI{1(1)}{\kilogram\per\meter\cubed}$ und der angegebenen Wärmekapazität der Apparatur $c_km_k = \SI{750(10)}{\joule\per\kelvin}$ kann dann die Güteziffer zu verschiedenen Zeitpunkten ermittelt werden.

  Im Vergleich dazu sollte nach (\ref{eqn:güte_ideal}) zu den selben Zeitpunkten die Güteziffer einer idealen Wärmepumpe ermittelt werden, die bei den entsprechenden Temperaturniveaus arbeitet.

  Die Ergebnisse sind Tabelle \ref{tab:ergebnisse_güte} und Abb. \ref{fig:temperatur} zu entnehmen.

  \begin{table}
    \centering
    \caption{Ermittelte Gütezahlen.}
    \label{tab:ergebnisse_güte}
    \sisetup{
      round-mode=places,
      round-precision=3,
    }
    \begin{tabular}{
        l@{}
        S[table-format=4.0]
        S[table-format=1.3] @{${}\pm{}$} S[table-format=1.3]
        S[table-format=2.3] @{${}\pm{}$} S[table-format=1.3]
        S[table-format=2.3] @{${}\pm{}$} S[table-format=1.3]}
      \toprule
      & $t / \si{\second}$
      & \multicolumn{2}{c}{$ν$}
      & \multicolumn{2}{c}{$ν_{\mathrm{ideal}}$}
      & \multicolumn{2}{c}{$ν / ν_{\mathrm{ideal}}$} \\
      %& \multicolumn{2}{c}{$\dot m / \si{\gram\per\second}$}
      %& \multicolumn{2}{c}{$N / \si{\watt}$} \\
      \midrule
      \input{table_guete.tex}
      \bottomrule
    \end{tabular}
  \end{table}

  \begin{figure}
    \centering
    \includegraphics{plot.pdf}
    \caption{Plot des Temperaturverlaufs und der Fits.}
    \label{fig:temperatur}
  \end{figure}

\subsection{Massendurchsatz}
  Außerdem soll der Massendurchsatz des Kreislaufes ermittelt werden. Dies geschieht, indem (\ref{eqn:durchsatz1}) und (\ref{eqn:durchsatz2}) zu
  \begin{equation}
    \frac{\mathrm{d}m}{\mathrm{d}t} = \dot m = \frac{(m_{2} c_w + m_k c_k)}{L}\frac {\mathrm{d}T_{2}}{\mathrm{d}t}
  \end{equation}
  umgestellt werden. Der Differentialquotient $\frac {\mathrm{d}T_{2}}{\mathrm{d}t}$ wird exakt wie in Abschnitt \ref{subsec:güteziffer} mittels nichtlinearer Ausgleichsrechnung bestimmt. Die Verdampfungswärme $L$ lässt sich aus der (vom Manometer abgelesenen) Dampfdruckkurve von Dichlordifluormethan (Tabelle \ref{tab:dampfdruck}) mithilfe einer linearen Regression des logarithmierten Drucks über die reziproke Temperatur ermitteln. Es gilt nach \cite{anleitung203}
  \begin{equation}
    \ln p = -\frac{L}{R} \frac {1}{T} + \mathrm{const.}
  \end{equation}
  Damit ergibt sich $L$ zu $-R \cdot a$ mit der Steigung $a$ der Ausgleichsgeraden. Eine grafische Darstellung der Dampfdruckkurve findet sich in Abb. \ref {fig:dampfdruck}, die Ergebnisse für den Massendurchsatz in Tabelle \ref{tab:ergebnisse_durchsatz}.

  \begin{table}
    \centering
    \caption{Ermittelte Massendurchsätze.}
    \label{tab:ergebnisse_durchsatz}
    \sisetup{
      round-mode=places,
      round-precision=3,
    }
    \begin{tabular}{
        l@{}
        S[table-format=4.0]
        S[table-format=1.3] @{${}\pm{}$} S[table-format=1.3]}
      \toprule
      & $t / \si{\second}$
      & \multicolumn{2}{c}{$\dot m$} \\
      \midrule
      \input{table_massendurchsatz.tex}
      \bottomrule
    \end{tabular}
  \end{table}

  \begin{figure}
    \centering
    \includegraphics{plot_dampfdruck.pdf}
    \caption{Plot der Dampfdruckkurve.}
    \label{fig:dampfdruck}
  \end{figure}

  \begin{table}
    \centering
    \caption{Dampfdruckkurve von Dichlordifluormethan.}
    \label{tab:dampfdruck}
    \begin{tabular}{l@{}
        S[table-format=3.2]
        S[table-format=4.0]
      }
      \toprule
      & $T / \si{\kelvin}$
      & $p/ \si{\kilo\pascal}$\\
      \midrule
      \input{table_dampfdruck.tex}
      \bottomrule
    \end{tabular}
  \end{table}

\subsection{Mechanische Kompressorleistung}
  Zuletzt sollte die menchanische Leistung des Kompressors ermittelt werden. Diese ist nach (\ref{eqn:mechleistung})
  \begin{equation}
    N_\mathrm{mech} = \frac{1}{\kappa-1}\biggl(p_b \sqrt[\kappa]{\frac{p_a}{p_b}}-p_a\biggr)\frac{1}{\rho} \frac{\mathrm{d}m} {\mathrm{d}t}.
  \end{equation}
  Dabei ist die Dichte $\rho$ mithilfe der allgemeinen Gasgleichung zu bestimmen:
  \begin{align}
    pV &=nRT \\
    \Leftrightarrow \frac{pV}{T} &= nR
    \intertext{Im geschlossenen System gilt: $nR = \mathrm{const.}$ Also ist }
    \frac{p_1 V_1}{T_1} = \frac{p_2 V_2}{T_2}
  \end{align}

\begin{table}
  \scriptsize
  \centering
  \caption{Messdaten.}
  \label{tab:messdaten}
  \begin{tabular}{l@{}
      S[table-format=4.0]
      S[table-format=3.2]
      S[table-format=3.2]
      S[table-format=3.0]
      S[table-format=4.0]
      S[table-format=3.0]
    }
    \toprule
    & $t / \si{\second}$
    & $T_{1} / \si{\kelvin}$
    & $T_{2} / \si{\kelvin}$
    & $p_a / \si{\kilo\pascal}$
    & $p_b / \si{\kilo\pascal}$
    & $P / \si{\watt}$ \\
    \midrule
    \input{table_data.tex}
    \bottomrule
  \end{tabular}
\end{table}
