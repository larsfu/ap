\section{Diskussion}
\label{sec:Diskussion}
Alle Werte der Messreihen, welche zur Bestimmung der Empfindlichkeit der Braunschen Röhre durchgeführt wurden, liegen auf der jeweiligen Ausgleichsgeraden.
Die ermittelten Empfindlichkeiten für die Braunsche Röhre zeigen untereinander kleine Abweichungen, was darauf hindeutet, dass die Messung nur geringe Ungenauigkeiten aufweist.
Das Auftragen der Empfindlichkeit $\frac{D}{U_\mathrm{d}}$ gegen $\frac{1}{U_\mathrm{B}}$ bestätigt dies, da die ermittelte Steigung lediglich um $(28 \pm 0,04)\%$ von der Apparaturkonstante abweicht.
Auch die Werte der Messreihen zur Bestimmung der spezifischen Ladung können durch die Ausgleichsgeraden beschrieben werden. Der gemittelte Wert der spezifischen Ladung $\langle\frac{e}{m_\mathrm{e}}\rangle=(1,78 \pm 0,04)10^{11} \frac{\si{\coulomb}}{\si{\kilo\gram}}$ weicht nur geringfügig vom Literaturwert $\frac{e_\mathrm{lit}}{m_\mathrm{e, lit}}=1,759 \cdot 10^
{11} \si{\coulomb\per\kilo\gram}$ \cite{formelsammlung} ab. Die Abweichung beträgt $(1,7 \pm 2,3)\%$.
Die  berechnte Horizontalkomponente des Erdmagnetfeldes $B_\mathrm{hor}=25,45 \cdot 10^{-6} \si{\tesla}$ weicht um 92,53\% vom Literaturwert $B_\mathrm{hor,lit}=49 \cdot 10^{-6} \si{\tesla}$ \cite{wolframalpha} ab. Die Abweichung kann darauf zurückgeführt werden, dass nur abgeschätzt werden konnte, wann der Leuchtfleck sich wieder in seiner ursprünglichen Lage befand.
