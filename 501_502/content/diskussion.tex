\section{Diskussion}
\label{sec:Diskussion}
Die ermittelten Empfindlichkeiten für die Braunsche Röhre zeigen untereinander kleine Abweichungen, was darauf hindeutet, dass die Messung nur geringe Ungenauigkeiten aufweist.
Das Auftragen der Empfindlichkeit $\frac{D}{U_\mathrm{d}}$ gegen $\frac{1}{B}$ bestätigt dies, da die ermittelte Steigung lediglich um $(4,7 \pm3)\%$ von der Apparaturkonstante abweicht.
Der gemittelte Wert der spezifischen Ladung $\frac{e}{m_\mathrm{e}}=(6,9 \pm 0,5)10^9 \frac{\si{\coulomb}}{\si{\kilo\gram}}$ zeigt eine große Abweichung vom Literaturwert $\frac{e_\mathrm{lit}}{m_\mathrm{e, lit}}=1,759 \cdot 10^
{11} \si{\coulomb\per\kilo\gram}$ \cite{tafelwerk}.
Die  berechnte Horizontalkomponente des Erdmagnetfeldes $B_\mathrm{hor}=25,45 \cdot 10^{-6} \si{\tesla}$ weicht um 27,25\% vom Literaturwert $B_\mathrm{hor,lit}=20 \cdot 10^{-6} \si{\tesla}$ \cite{dornbader} ab. Die Abweichung kann darauf zurückgeführt werden, dass nur abgeschätzt werden konnte, wann der Leuchtfleck sich wieder in seiner ursprünglichen Lage befand.
