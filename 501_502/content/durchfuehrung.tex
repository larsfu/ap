\section {Aufbau und Durchführung}
\label{sec:durchführung}

\subsection{Elektrisches Feld}
Zur Bestimmung der Ablenkung durch ein elektrisches Feld wird eine Kathodenstrahlröhre, wie in Abb.~\ref{fig:kath} dargestellt, verwendet. Die im Vakuum thermisch emittierten Elektronen werden durch einen Wehneltzylinder, mit dem Intensität reguliert werden kann und durch eine davor angebrachte gelochte Elektrode beschleunigt, sowie durch Fokussierungselektroden auf den Leuchtschirm fokussiert. Es folgen zwei Ablenkelektroden, jeweils für eine der beiden zueinander senkrechten Raumrichtungen.

\fig{bilder/1}{Querschnitt einer Kathodenstrahlröhre. \cite{anleitung501502}}{kath}

Mithilfe der in Abb.~\ref{fig:sch1} dargestellten Schaltung kann die Verschiebung des Leuchtflecks in Abhängigkeit von der Beschleunigungs- und Ablenkspannung bestimmt werden. Daraus lässt sich dann die Empfindlichkeit $\sfrac{d}{U_d}$ aus aus \eqref{eqn:prop} bestimmen. Weiterhin wird im zweiten Versuchsteil als $x$-Ablenkung eine Sägezahnspannung angelegt, um den zeitlichen Verlauf der an die $y$-Elektrode angeschlossenen Spannungsquelle beobachten zu können. Dies erfolgt durch die Schaltung in Abb.~\ref{fig:sch2}. Damit auf dem Schirm eine \enquote{stehende Welle} sichtbar wird, muss die Synchronisationsbedingung
\begin{equation}
  n f_\text{Sägezahn} = m f_\text{Schwingung}, \quad n \in ℕ
\end{equation}
erfüllt werden. Daher muss die Sägezahnfrequenz entsprechend eingestellt werden. Die Amplitude der Spannung kann Anhand der Ablenkung auf dem Schirm abgemessen werden.

\fig{bilder/3}{Schaltung zur Bestimmung der Empfindlichkeit. \cite{anleitung501502}}{sch1}
\fig{bilder/4}{Schaltung als Oszillograph. \cite{anleitung501502}}{sch2}

\subsection{Magnetisches Feld}
Der Versuchsaufbau ist ähnlich wie beim elektrischen Feld, jedoch wird die Kathodenstrahlröhre in ein durch ein Helmholtzspulenpaar erzeugtes homogenes Magnetfeld eingebracht.
Zuerst wird der Aufbau so gedreht, dass die Horizontalkomponente des Erdmagnetfelds parallel zum Versuchsaufbau steht. Dann wird die Spule eingeschaltet und erzeugt ein Magnetfeld
\begin{equation}
  \label{eqn:spule}
  B = \mu_0 \frac{8}{\sqrt{125}} \frac{NI}{R},
\end{equation}
mit dem Spulenradius
\begin{equation}
  R = \SI{28.2}{cm}
\end{equation}
und der Windungszahl
\begin{equation}
  N = 20.
\end{equation}
Die Beschleunigungsspannung sowie der Spulenstrom werden variiert und die Ablenkung des Leuchtpunktes wird gemessen. So kann nach \eqref{eqn:spez} die spezifische Ladung des Elektrons $\sfrac{e}{m_e}$ bestimmt werden.

Im zweiten Versuchsteil wird das Erdmagnetfeld dadurch bestimmt, dass der Aufbau nun senkrecht zur Horizontalkomponente des Erdmagnetfelds gestellt wird und dann die erzeugte Ablenkung dann durch das \enquote{künstliche} Magnetfeld kompensiert wird. Da hierbei nur die horizontale Komponente gemessen wird, wird zur Bestimmung der Totalintensität  daraufhin noch der Inklinationswinkel $\phi$ des Erdmagnetfelds mit einem Kompass ausgemessen.
