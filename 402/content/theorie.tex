\section{Ziel}
\label{sec:Ziel}
In diesem Experiment soll der Brechungsindex $n$ in Abhängigkeit von der Wellenlänge $\lambda$ (Dispersion) an einem Glasprisma untersucht werden. Zunächst soll die Dispersionsgleichung bestimmt werden. Damit können die Abbesche Zahl $\nu$, das Auflösungsvermögen $A$ des Spektralapparates und die zum sichtbaren Licht am nächsten gelegene Absorptionsstelle $\lambda_i$ berechnet werden.

\section{Theorie}
\label{sec:theorie}
\subsection{Brechung}
Trifft ein Lichtstrahl auf die Grenzfläche zweier optischer Medien, wird dieser gebrochen. Der Brechungsindex $n$, welcher spezifisch für das jeweilige Medium ist, ergibt sich aus folgendem Zusammenhang:
\begin{equation}
  \label{eqn:n1}
  n=\frac{v_1}{v_2}.
\end{equation}
Dabei sind $v_1$ und $v_2$ die Geschwindigkeiten des Lichts in den beiden Medien.

\subsubsection{Huygenssches Prinzip}
Um eine Aussage über die Richtungsänderung und die Geschwindigkeit des Strahls treffen zu können, wird das Huygenssche Prinzip betrachtet. Es besagt, dass von jedem Punkt einer Wellenfront eine kugelförmige Elementarwelle ausgeht. Somit ist jede Wellenfront die Einhüllende aller Elementarwellen. Mit dem Brechungsgesetz
\begin{equation}
  \frac{\sin(\alpha)}{\sin(\beta)} =\frac{v_1}{v_2}
\end{equation}
mit $\alpha$ als Winkel vor und $\beta$ als Winkel nach der Brechung an der Grenzfläche und Gleichung \ref{eqn:n1} folgt
\begin{equation}
  \frac{\sin(\alpha)}{\sin(\beta)} = n.
\end{equation}

\subsection{Dispersionsgleichung}
Da es sich bei Licht um eine elektromagnetische Welle handelt, ist es in der Lage die sich im Material befindenden Elektronen und Ionenrümpfe zu Schwingungen anzuregen. Die Polarisation $\vec{P}$ der Materie ist gegenüber der elektrischen Feldstärke $\vec{E}$ der Lichtwelle vernachlässigbar. Durch das magnetische Wechselfeld der Welle wirkt eine Lorentzkraft auf die in der Materie befindlichen Elektronen und Ionenrümpfe, welche diese um $\vec{x}_\mathrm{h}$ verschiebt, wodurch ein elektrischer Dipol entsteht. Des Weiteren wirkt eine rücktreibende Kraft, welche proportional zur Auslenkung ist. Außerdem wird die Schwingung der Teilchen durch eine weitere Kraft gedämpft, welche proportional zu Geschwindigkeit der Elektronen und Ionenrümpfe ist. Mit diesen Kräften kann eine Differentialgleichung für die Bewegung der Teilchen aufgestellt werden:
\begin{equation}
  m_\mathrm{h} \frac{\mathrm{d}\vec{x}_\mathrm{h}}{\mathrm{d}t^2}+f_\mathrm{h}\frac{\mathrm{d}\vec{x}_\mathrm{h}}{\mathrm{d}t}+a_\mathrm{h}\,\vec{x}_\mathrm{h} = q_\mathrm{h}\vec{E}_0 e^{i\omega t}.
\end{equation}
Daber entspricht $m_\mathrm{h}$ den Teilchenmassen, $d_\mathrm{h}=q_\mathrm{h}\,\vec{x}_\mathrm{h}$ dem Dipolmoment ($q_\mathrm{h}$ ist Summe der Teilchen pro Volumeneinheit), $f_\mathrm{h}$ dem Dämpfungsfaktor und $a_\mathrm{h}$ dem Proportionaliätsfakor der rücktreibenden Kraft.
Mit Hilfe von
\begin{equation}
  \vec{P} = \sum N_\mathrm{h}\,q_\mathrm{h}\,\vec{x}_\mathrm{h}
\end{equation}
und $n^2 =\epsilon$ folgt:
\begin{equation}
\tilde{n}^2=1+\sum_{h}\frac{1}{\omega_\mathrm{h}^2-\omega^2+i\frac{f_\mathrm{h}}{m_\mathrm{h}}\.\omega\,m_\mathrm{h}\,\epsilon_0}
\end{equation}
mit $\omega_\mathrm{h}$ als Resonanzfrequenz des Systems. $\tilde{n}$ ist komplex.
Es wird nur der Bereich $n^2\,k \approx 0$ betrachtet, da die Gleichung für andere Fälle nicht zutrifft. Dann gilt für den Brechungsindex $n$:
\begin{equation}
  n^2(\lambda)=1+\sum \frac{N_\mathrm{h}\,q_\mathrm{h}^2}{4\pi^2\,c^2\,\epsilon_0\,m_\mathrm{h}}\frac{\lambda^2\,\lambda_\mathrm{h}^2}{\lambda^2-\lambda_\mathrm{h}^2}
\end{equation}

\subsubsection{Normale Dispersion}
Bei der normalen Dispersion nimmt der Brechungsindex mit zunehmender Wellenlänge ab, wie in Abbildung \ref{fig:normal} dargestellt. Sie wird durch folgende Gleichung beschrieben:
\begin{equation}
  n^2(\lambda)=A_0+\frac{A_2}{\lambda^2}+...
\end{equation}
Die Absorptionsstelle $\lambda_i < \lambda$ liegt im ultravioletten Bereich.
\fig{bilder/normal.pdf}{Dispersionskurve für normale Dispersion.\cite{anleitung402}}{normal}

\subsubsection{Anomale Dispersion}
Im Gegensatz zur normalen Dispersion fällt der Brechungsindex mit steigender Wellenlänge, was in Abbildung \ref{fig:anomal} graphisch dargestellt ist.
Die Dispersionskurve wird durch folgenden Zusammenhang beschrieben:
\begin{equation}
  n^2(\lambda)=A_0'-A_2'\lambda^2-...
\end{equation}
Für die Absoprtionsstelle gilt $\lambda_i>\lambda$. Sie befindet sich somit im Infraroten.
\fig{bilder/anomal.pdf}{Dispersionskurve für anomale Dispersion.\cite{anleitung402}}{anomal}

\subsection{Auflösungsvermögen eines Prismen-Spektralapparates}
Das Auflösungsvermögen $A$ beschreibt den kleinstmöglichen Abstand zwei nebeneinanderliegender Spektrallinien $\Delta \lambda$, bei dem Der Prismen-Spektralapparat in der Lage ist, diese zu trennen.
\begin{equation}
  \label{eqn:A1}
  A=\frac{\lambda}{\Delta \lambda}
\end{equation}
Weiterhin gilt:
\begin{equation}
  \label{eqn:A2}
 A=b \frac{\mathrm{d}n(\lambda)}{\mathrm{d}\lambda}.
\end{equation}
Dabei ist $b$ die Basisbreite des Prismas.
