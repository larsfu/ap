\section {Aufbau und Durchführung}
\label{sec:durchführung}
Das Experiment wird mit Zink und Kupfer durchgeführt. Zunächst werden die Abmessungen der beiden Proben bestimmt.
\subsection{Messung des elektrischen Widerstands}
Um den Widerstand der Proben zu messen, wird der Aufbau aus Abbildung \ref{fig:aufbau} verwendet.
\fig{bilder/aufbau.pdf}{Versuchsaufbau zur Messung der Widerstände. \cite{anleitung311}}{aufbau}
Sowohl für Zink als auch für Kupfer werden zu jeweils 12 Stromstärken im Bereich von 0 bis 5 \si{\ampere} und die entsprechenden Spannungen aufgenommen.
\subsection{Hall-Effekt}
Zur Erzeugung des Magnetfeldes dienen zwei Spulen, in deren Zwischenraum die Probe platziert wird. An den Punkten A und B der Probe kann die Hallspannung $U_\mathrm{H}$ abgegriffen werden.
Für Zink wird bei der Messung die Stromstärke $I$ im bereich von 0-8,0 \si{\ampere} stets um 0,5 \si{\ampere} erhöht und die jeweilige Hallspannung gemessen. Die Messung für Kupfer erfolgt analog mit einer Stromstärke im Bereich von 0-10 \si{\ampere}.
\subsection{Hysteresekurve}
Um die Hysteresekurve des vorliegenden Elektromagneten darstellen zu können, wird die Flussdichte $B$ in Abhängigkeit vom Feldstrom gemessen. Dafür befindet sich eine Hall-Sonde im $B$-Feld der Spulen. Die Stromstärke wird zunächst in Abständen von jeweils 0,5\si{\ampere} auf 5\si{\ampere} erhöht und der mittels Hall-Sonde gemessene Wert für die magnetische Flussdichte notiert. Dies wird für eine sinkende Stromstärke wiederholt.
