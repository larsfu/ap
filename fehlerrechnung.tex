\subsection{Fehlerrechnung}
\label{subsec:fehlerrechnung}

\subsubsection{Mittelwert}
\begin{equation}
\overline{v_i} = \frac{1}{n} \sum_{i=1}^n x_i
\end{equation}

\subsubsection{Standardabweichung}
\begin{equation}
s_i = \sqrt{\frac{1}{N - 1} \sum_{j=1}^N \left(v_j - \overline{v_i}\right){^2}}
\end{equation}

wobei $v_j$ ($j = 1, ..., N$) zufällige Fehler beschreibt.

\subsubsection{Streuung der Mittelwerte}
\begin{equation}
\sigma_i = \frac{s_i}{\sqrt{N}} = \sqrt{\frac{\sum_{j=1}^N \left(v_j - \overline{v_i}\right){^2}}{N \left(N - 1 \right)}}
\end{equation}

\subsubsection{Gaußfehler}
Hat man eine fehlerbehaftete Funktion $f$ mit $k$ als fehlerbehafteter Größe und $\sigma_k$ als Ungenauigkeit, gilt:

\begin{equation}
\Delta x_k = \frac{\mathrm{d}f}{\mathrm{d}k}\sigma_k
\end{equation}

Relativer Gaußfehler
\begin{equation}
\Delta x_\text{k, rel} = 1 \stackrel{+}{-} \frac{\Delta x_k}{|x|}\cdot 100\%
\end{equation}

Absoluter Gaußfehler
\begin{equation}
\Delta x_i = \sqrt{\left(\frac{\mathrm{d}f}{\mathrm{d}k_{1}}\cdot \sigma_{k_{1}}\right)^2 + \left(\frac{\mathrm{d}f}{\mathrm{d}k_{2}}\cdot \sigma_{k_{2}}\right)^2 + ...}
\end{equation}

\subsubsection{Lineare Regression}
\begin{equation}
\sigma {^2} = \sum_{k=1}{^N} \left(y_k - \left(\frac{\overline{xy} - \overline{x}\cdot\overline{y}}{\overline{x^2} - (\overline{x}){^2}}x_k + \left(\overline{y} - \overline{B}\overline{x}\overline{B}\right)\right)\right){^2}
\end{equation}
