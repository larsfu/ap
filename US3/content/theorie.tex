\section{Ziel}
Ziel des folgenden Experiments ist, typische Eigenschaften verschiedener Strömungen mittels Impuls-Echo-Verfahren zu bestimmen.

\section{Theorie}
\label{sec:theorie}
\subsection{Schall}
Schall kann anhand seiner Freqeunz in zwei Bereiche unterteilt werden.
\begin{enumerate}
  \item <16 \si{\Hz}: Infraschall
  \item 20 \si{\kilo\Hz} - 1\si{\giga\Hz}: Ultraschall
\end{enumerate}
Der für Menschen hörbare Schall befindet sich im Bereich von $16\si{\Hz}$ - $20\si{\kilo\Hz}$.

\subsection{Piezoelektrischer Effekt}
Im Allgemeinen wird unter dem piezoelektrischen Effekt verstanden, dass durch Verformung eines Kristalls eine Spannung auftritt. Der reziproke piezoelektrische Effekt bezeichnet den Vorgang, dass es durch eine Spannung (Kristall im elektrischen Wechselfeld) zu Verformungen kommt.

Ultraschall wird durch hier durch den reziproken piezo-elektrischen Effekt hervorgerufen. Dabei befindet sich ein piezoelektrischer Kristall mit einer polaren Achse in Feldrichtung in einem elektrischen Wechselfeld. Der Kristall beginnt zu schwingen und sendet dabei Ultraschallwellen aus. Wird ein piezoelektrischer Kristall als Schallempfänger verwendet, wird dieser von auftreffenden Schallwellen zum Schwingen angeregt.

\subsection{Doppler-Effekt}
Bewegen sich die Schallquelle und der Empfänger relativ zueinandern, tritt eine Frequenzänderung auf. Dies wird als Doppler-Effekt bezeichnet.
Zunächst wird der Fall mit ruhendem Empfänger und sich bewegender Quelle betrachtet. Wenn sich die Quelle in Richtung des Empfängers bewegt, wird die Frequenz $f_0$ zu $f_\mathrm{kl}$ erhöht und es gilt folgender Zusammenhang:
\begin{equation}
  f_\mathrm{kl} = \frac{f_0}{1-\frac{v}{c}}
\end{equation}
Dabei ist $v$ die Geschwindigkeit der Quelle und $c$ die Schallgeschwindigkeit.
Bewegt sich der Sender vom Empfänger weg, fällt die Frequenz auf $f_\mathrm{gr}$.
\begin{equation}
  f_\mathrm{gr} = \frac{f_0}{1+\frac{v}{c}}
\end{equation}
Im Folgenden bewegt sich der Empfänger und der Sender befindet sich in Ruhe. Wenn sich der Empfänger der Quelle nähert, verschiebt sich $f_0$ zu einer größeren Frequenz $f_\mathrm{h}$
\begin{equation}
  f_\mathrm{h} = f_0\left(1+\frac{v}{c}\right)
\end{equation}
mit  $v$ als Geschwindigkeit des Empfängers und $c$ als Schallgeschwindigkeit.
Die Frequenz sinkt auf eine kleinere Frequenz $f_\mathrm{n}$, wenn die Entfernung zwischen Sender und Empfänger durch die Bewegung des Empfängers steigt.
\begin{equation}
  f_\mathrm{n} =f_0\left(1-\frac{v}{c}\right)
\end{equation}

Mit Hilfe des beschriebenen Doppler-Effekts kann die Geschwindigkeit von Blutströmungen bestimmt werden. Beim Auftreffem auf sich bewegende Blutkörper wird die Frequenz $f_0$ der Ultraschallwelle wie folgt verschoben:
\begin{equation}
  \Delta f = f_0 \frac{v}{c}(\cos\alpha + \cos\beta)
\end{equation}
mit $\alpha$ als Winkel zwischen der Wellennormalen und der einlaufenden Welle und $\beta$ als Winkel zwischen der Wellennormalen und der auslaufenden Welle. Da beim Impuls-Echo-Verfahren $\alpha = \beta$ (siehe Abbildung \ref{fig:1}) gilt, folgt:
\begin{equation}
  \Delta f = 2\,f_0\frac{v}{c}\cos\alpha
\end{equation}

\begin{figure}
  \centering
  \includegraphics[scale=0.8]{content/1.pdf}
\caption{Messung der Geschwindigkeit von Blutströmungen mittels Impuls-Echo-Verfahren\cite{anleitungUS3}.}
  \label{fig:1}
\end{figure}
