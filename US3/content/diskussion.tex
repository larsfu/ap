\section{Diskussion}
\label{sec:Diskussion}

Die Strömungsgeschwindigkeit kann im ersten Versuchsteil mit geringer Messabweichung zwischen den verschiendene Dopplerwinkeln ermittelt werden. Die Durchflussgeschwindigkeit ist wie erwartet bei kleineren Rohrdurchmessern höherem Volumenstrom höher.
Im Strömungsprofil kann zwischen \SI{33}{mm} und \SI{42}{mm}, also innerhalb des Rohres, ein annähernd laminarer Geschwindigkeitsverlauf gefunden werden. Außerhalb dieses Bereichs ist die Geschwindigkeit konstant, obwohl eigentlich gar keine Bewegung zu erwarten wäre. Dies hat vermutlich messtechnische Gründe. In diesem Bereich ist bei der Intensitätsverteilung ein linear steigender Verlauf zu beobachten.
