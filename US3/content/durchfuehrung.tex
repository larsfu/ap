\section {Aufbau und Durchführung}
\label{sec:durchführung}

Der Aufbau setzt sich aus einem Ultraschall Doppler-Generator sowie Ultraschallsonden mit einer Frequenz von 2\si{\mega\Hz} zusammen. Um die Messdaten aufzunehmen, wird ein Computer verwendet. Desweiteren besteht der Aufbau aus Strömungsrohren mit drei verschiedenen Innen- und Außendurchmessern (7\si{\milli\meter}, 10\si{\milli\meter}, 16\si{\milli\meter}) und zum Rohrdurchmesser passenden Doppler-Prismen. Ein Doppler-Prisma, welches in Abbildung \ref{fig:prisma} dargestellt ist, besitzt drei verschiedene Einfallswinkel ($\theta = 15^\circ, 30^\circ, 45^\circ$). Der Dopplerwinkel $\alpha$ lässt sich mit folgendem Zusammenhang berechnen:
\begin{equation}
   \alpha = 90^\circ -arcsin \left(\sin\theta\cdot\frac{c_\mathrm{L}}{c_\mathrm{P}}\right)
\end{equation}
mit $c_\mathrm{L}$ als Schallgeschwindigkeit Doppler-Flüssigkeit und $c_\mathrm{P}$ als Schallgeschwindigkeit des Materials des Prismas. Die Prismen bestehen aus Acryl.
Die verwendete Flüssigkeit besteht aus Wasser, Glycerin und Glaskugeln. Mit Hilfe einer Zentrifugalpumpe kann die Strömungsgeschwindigkeit zwischen 0 und 10\si{\liter\per\minute} variiert werden.

\begin{figure}
  \centering
  \includegraphics{content/prisma.pdf}
\caption{Doppler-Prisma\cite{anleitungUS3}.}
  \label{fig:prisma}
\end{figure}

Zunächst wird der Dopplerwinkel für fünf Strömungsgeschwindigkeiten gemessen. Dafür wird bei einer eingestellten Strömungsgeschwindigkeitbei jedem Winkel des Prismas die Frequenzverschiebung $\Delta f$ gemessen. Dies wird für vier weitere Geschwindigkeiten wiederholt. Die gleiche Messung wird erneut mit den beiden anderen Prismen durchgeführt.

Um das Strömungsprofil der Doppler-Flüssigkeit am 10\si{\milli\meter} Rohr unter $\alpha=15^\circ$ zu untersuchen, wird die Messtiefe variiert. Dafür wird am Ultraschallgenerator das SAMPLE VOLUME aud SMALL gestellt. Bei der Einheit der Messtiefe handelt es sich hier um $\textmu\si{\second}$. Diese wird im Bereich von 4 bis 19,5 $\textmu\si{\second}$ stetig um 0,5$\textmu\si{\second}$ erhöht und der entsprechende Wert für die Frequenzverschiebung $\Delta f$ sowie für die Streuintensität $I$ aufgenommen. Dies wird für eine Pumpleistung von $45\%$ sowie $75\%$ durchgeführt.

In Acryl gilt $4\textmu\si{\second}=10\si{\milli\meter}$. Für die Dopplerflüssigkeit gilt $4\textmu\si{\second}=6\si{\milli\meter}$.
