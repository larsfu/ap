\section{Diskussion}
\label{sec:Diskussion}
Die Abweichungen werden über
\begin{equation}
  \Delta d^\text{rel} = \frac{|d - d^\text{mech}|}{d^\text{mech}}
\end{equation}
bestimmt und finden sich in Tab.~\ref{tab:diska} und~\ref{tab:diskb}. Wie dort zu erkennen ist, konnten die Maße der Störstellen größtenteils mit geringem Fehler bestimmt werden, wobei der Durchmesser aus ersichtlichen Gründen einen größeren relativen Fehler besitzt. Die Störstellen 1 und 2, die zur Bestimmung des Auflösungsvermögens dienen, sind besser durch den B-Scan als durch den A-Scan ermittelt worden, insgesamt liegen die Abweichungen der beiden Messungen jedoch ähnlich.
\begin{table}
        \caption{Abweichungen beim A-Scan.}
        \centering
        \label{tab:diska}
        \begin{tabular}{l@{}S[round-mode=off, table-format=2.0]S[table-format=2.3, round-precision=2, round-mode=places] S[table-format=1.3, round-precision=2, round-mode=places] S[table-format=2.1, round-precision=2, round-mode=places] } \toprule & {$\text{Störstelle}$}& {$\Delta d_1/\si{\percent}$}& {$\Delta d_2/\si{\percent}$}& {$\Delta 2r/\si{\percent}$}\\\midrule& 1 & 8.16513761467891363566 & 3.25645756457564372610 & 42.43749999999998578915 \\
& 2 & 12.21283783783782084242 & 2.46818923327896788678 & 66.93877551020337079990 \\
& 3 & 0.01387075718013655937 & 2.57812500000001021405 & 6.67293233082708248105 \\
& 4 & 0.39747399702821684109 & 1.90322580645159766810 & 13.93333333333320744885 \\
& 5 & 0.39354838709679473840 & 1.19514627659575189966 & 15.67919075144557794488 \\
& 6 & 1.13730569948187021367 & 0.82386363636365389507 & 27.86764705882399795200 \\
& 7 & 0.97988319273198154491 & 0.45328498293513319606 & 21.98717948717927583857 \\
& 8 & 1.26646180860401047497 & 0.59467347683327231866 & 25.18442622950816911498 \\
& 9 & 1.31578947368420373643 & 0.08494760241343937701 & 6.31696428571477053282 \\
& 10 & 5.15261627906975139268 & 0.25423728813559265394 & 22.64830508474621240111 \\
& 11 & 1.28537735849056766746 & 5.43062200956937779495 & 14.60641399416919483656 \\
 \bottomrule \end{tabular} \end{table}

\begin{table}
        \caption{Mittelwerte der Messungen mit A- und B-Scan.}
        \centering
        \label{tab:diskb}
        \begin{tabular}{l@{}S[round-mode=off, table-format=2.0]S[table-format=2.0, round-precision=0, round-mode=places] @{${}\pm{}$} S[table-format=1.0, round-precision=0, round-mode=places] S[table-format=2.0, round-precision=0, round-mode=places] @{${}\pm{}$} S[table-format=1.0, round-precision=0, round-mode=places] S[table-format=2.0, round-precision=0, round-mode=places] @{${}\pm{}$} S[table-format=1.0, round-precision=0, round-mode=places] } \toprule & {$\text{Störstelle}$}& \multicolumn{2}{c}{$d_1/\si{\mm}$}& \multicolumn{2}{c}{$d_2/\si{mm}$}& \multicolumn{2}{c}{$2r/\si{mm}$}\\\midrule& 1 & 16.72125000000000127898 & 1.29674999999999940314 & 58.62674999999999414513 & 2.93475000000000285638 & 4.69200000000001526956 & 3.20847598167728431662 \\
& 2 & 16.65300000000000224532 & 3.27599999999999846878 & 58.42199999999998993871 & 1.36500000000000154543 & 4.96500000000001140421 & 3.54899999999999904432 \\
& 3 & 59.17274999999998641442 & 2.11574999999999935341 & 11.53425000000000011369 & 2.25225000000000097344 & 9.33300000000001084288 & 3.09015017838939343164 \\
& 4 & 51.80174999999999130296 & 2.25225000000000408207 & 19.79249999999999687361 & 2.32049999999999956302 & 8.44575000000000919442 & 3.23378266315162532507 \\
& 5 & 44.36249999999999715783 & 2.32050000000000666844 & 28.05075000000000073896 & 2.38875000000000037303 & 7.62675000000000125056 & 3.33029230136035225840 \\
& 6 & 36.99149999999999494094 & 2.04750000000000387246 & 36.78674999999999783995 & 2.25225000000000408207 & 6.26175000000000370193 & 3.04382757601346964904 \\
& 7 & 28.73324999999999462830 & 2.38875000000000037303 & 44.90850000000000363798 & 2.18399999999999483435 & 6.39825000000000088107 & 3.23666225647656657927 \\
& 8 & 21.08924999999999627676 & 1.97924999999999839950 & 52.68900000000000005684 & 2.45700000000000429168 & 6.26175000000000370193 & 3.15504034245206055331 \\
& 9 & 12.83099999999999951683 & 2.18399999999999927525 & 60.67425000000000068212 & 2.25225000000000763478 & 6.53475000000000605382 & 3.13727366713521371722 \\
& 10 & 4.98224999999999873523 & 2.25224999999999964118 & 69.13724999999999454303 & 1.84275000000000410694 & 5.92050000000000942180 & 2.91004426512725311582 \\
& 11 & 53.71274999999999977263 & 2.11574999999999935341 & 13.03575000000000017053 & 2.38874999999999992895 & 13.29150000000000275691 & 3.19100683562414122463 \\
 \bottomrule \end{tabular} \end{table}


Bezüglich des TM-Scans kann festgestellt werden, dass die Frequenz, mit der das Herzmodell bewegt wurde mit
\begin{equation*}
  \nu_\text{Herz} = \SI{38.4}{\per\minute}
\end{equation*}
relativ gering liegt, was allerdings durch die Bedienung des Experimentators zu begründen ist. Schlag- und Herzminutenvolumen scheinen plausibel, allerdings liegen keine Vergleichswerte aus der Literatur vor.
