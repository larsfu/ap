\section{Diskussion}
\label{sec:Diskussion}
Die Viskosität kann jeweils mit einer sehr geringen Messungenauigkeit (ca. \SI{1}{\%}) bestimmt werden. Allerdings fällt beim Vergleich mit interpolierten (siehe Abb.~\ref{fig:etaref}) Literaturwerten auf, dass sie jeweils um ca. $\SI{35}{\%}$ zu groß bestimmt wurde (siehe Tabelle~\ref{tab:vergl}). Mögliche Ursachen sind die Reibung der Kugel an der Rohrwand, sowie kleine Luftblasen im Aufbau. Eine fehlerhafte Messung durch turbulente Strömungen ist nicht zu erwarten, da die höchste ermittelte Reynolds-Zahl bei $Re = \num{61.9+-0.8}$ und damit deutlich kleiner als $Re_\text{krit} = 2000$, ab der Turbulenzen zu erwarten sind.

\begin{table}
  \caption{Vergleich der Messwerte mit interpolierten Literaturwerten aus \cite{wärmeatlas}.}
  \centering
  \label{tab:vergl}
  \begin{tabular}{l@{} S
      S[table-format=1.3, round-precision=3, round-mode=figures] @{${}\pm{}$} S[table-format=1.3, round-precision=1, round-mode=figures]
      S[table-format=1.3, round-precision=3, round-mode=figures]
      S[table-format=1.2, round-precision=3, round-mode=figures] @{${}\pm{}$} S[table-format=1.3, round-precision=1, round-mode=figures]}
    \toprule
     &{$T/\si{K}$} & \multicolumn{2}{c}{$\eta/\si{\milli\pascal\second}$} & {$\eta_\text{ref}/\si{\milli\pascal\second}$} &\multicolumn{2}{c}{$\eta/\eta_\text{ref}$} \\
    \midrule
    \input{build/disk.tex}
    \bottomrule
  \end{tabular}
\end{table}

\fig{build/eta}{Interpolation der Literaturwerte aus \cite{wärmeatlas} mittels einer quadratischen Funktion, um einen Vergleich mit den Messwerten zu ermöglichen.}{etaref}
