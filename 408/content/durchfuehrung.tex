\section {Aufbau und Durchführung}
\label{sec:durchführung}
\subsection{Messung der Gegenstandsweite und Bildweite}
Bei fester Gegenstandsweite $g$ wird der Schirm so lange verschoben, bis auf diesem ein scharfes Bild erkennbar ist. Dies wird für 10 verschiedene Gegenstandsweiten wiederholt und dazugehörige Bildweite $b$ wird aufgenommen.

\subsection{Methode von Bessel}
\fig{bilder/bessel.pdf}{Schematischer Aufbau zur Bestimmung der Brennweite mit der Methode von Bessel \cite{anleitung408}.}{Bessel}
In Abbildung \ref{fig:Bessel} ist der Aufbau für die Methode von Bessel schematisch dargstellt. Hier ist der Abstand $e$ zwischen Gegenstand und Schirm fest. Die Position der Linse wird so verändert, dass auf dem Schirm ein scharfes Bild zu sehen ist. Die Werte für die Bildweite $b_1$ und die Gegenstandeite $g_1$ werden notiert. Dann wird die Linse erneut verschoben, bis ein scharfes Bild auf dem Schirm erkennbar ist. Erneut werden die zugehörigen Werte $b_2$ und $g_2$ aufgenommen.
Es gelten folgende Zusammenhänge:
\begin{align}
  b_1 &= g_2 \\
  b_2 &= g_1 \\
  e &= g_1+b_1 =g_2+b_2 \\
  d &= g_1 - b_1 = g_2 -b_2
\end{align}
Daraus lässt sich die Brennweite $f$ berechnen:
\begin{equation}
  f = \frac{e^2 - d^2}{4e}
\end{equation}

Die Messung wird für insgesamt 10 verschiedene Abstände $e$ zwischen Gegenstand und Schirm durchgeführt.

Zusätzlich soll die chromatische Aberration betrachtet werden. Dazu wird einmal ein blauer und einmal ein roter Filter eingesetzt und der zuvor erläuterte Messvorgang für jeweils 5 Abstände $e$ zwischen Gegenstand und Schirm durchgeführt.

\subsection{Methode von Abbe}
\fig{bilder/abbe.pdf}{Schematischer Aufbau zur Bestimmung der Brennweite mit der Methode von Abbe \cite{anleitung408}.}{abbe}
Wie Abbildung \ref{fig:abbe} zeigt, besteht der Aufbau aus einer Sammel- und einer Zerstreuungslinse. Hier soll die Brennweite des Linsensystems ermittelt werden. Über den Abbildungsmaßstab $V$ kann sowohl die Brennweite als auch die Lage der Hauptebenen bestimmt werden. Es wird ein beliebiger Referenzpunkt $A$ ausgesucht, weil nicht bekannt ist, wo die Hauptebenen liegen. Die Längen $b'$ und $g'$ sind die gemessene Bild- beziehungsweise Gegenstandsweite. Damit kann der Brennpunkt und die Position der Hauptebenen ermittelt werden.
\begin{align}
  g' &= g + h = f \cdot\left( 1+\frac{1}{V}\right)+h\\
  b' &= b + h' = f \cdot (1 + V) + h'
\end{align}

Die Messung wird für 10 Gegenstandsweiten durchgeführt.
