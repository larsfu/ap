\section{Diskussion}
\label{sec:Diskussion}
Da kein nomineller Wert für $RC$ gegeben ist, kann kein Vergleich damit zur Evaluation der Messdaten angestellt werden, es können lediglich die einzelnen Messungen verglichen werden. Eventuelle Ungenauigkeiten der Methode in Abschnitt \ref{sec:a} liegen etwa darin, dass die Rechteckspannung nicht optimal ist und z.\,B. Überschwinger (Gibb'sches Phänomen) aufweist. Andererseits ist die Messung durch die Messunsicherheit bzw. Genauigkeit des Oszilloskops beschränkt, was auch in Abb. \ref{fig:auswertung_a} an den \enquote{Stufen} im niedrigen Spannungsbereich zu sehen ist.
Bei den Sinus-gespeisten Messungen in \ref{sec:b} ist problematisch, dass eine eventuelle Fehlerbehaftung der Frequenz bei der Auswertung nicht berücksichtigt wird. Wenn die ermittelten $RC$-Werte verglichen werden fallen Abweichungen auf, diese werden eventuell durch den Innenwiderstand des Spannungsgenerators verursacht, der in der Auswertung nicht berücksichtigt wurde.

Die Verläufe der frequenzabhängigen Größen $\phi$ und $A$ entsprechen den Erwartungen.
