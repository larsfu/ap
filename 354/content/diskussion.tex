\section{Diskussion}
\label{sec:Diskussion}

Die Abweichung des Dämpfungswiderstandes $R_\text{eff}$ vom erwarteten Wert lässt sich durch nicht beachtete Bauteilwiderstände erklären, denn außer $R_1$ und $R_\text{Generator}$ wurden alle Bauteile als ideal angenommen.
Beim Grenzwiderstand $R_\text{ap}$ ist die Abweichung mit $\SI{33,9}{\%}$ jedoch zu groß um durch dieselbe Begründung erklärbar zu sein. Eine mögliche Fehlerquelle ist dabei jedoch der Messvorgang, da der aperiodische Grenzfall durch Hinschauen ermittelt wird.
Die Amplitude bei der erzwungenen Schwingung weist im niedrigen Frequenzbereich einen unerwartet hohen Wert auf, welcher durch unberücksichtigte Bauteileigenschaften zu erklären ist. Resonanzüberhöhung und Resonanzkurvenbreite liegen nah an den berechneten Werten, dies kann wiederum durch unberücksichtigte Widerstände begründet werden.
Resonanzfrequenz und $\nu_{1,2}$ sind genau bestimmt, die minimalen Abweichungen sind durch Bauteiltoleranzen abgedeckt.
