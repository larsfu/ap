\section{Diskussion}
\label{sec:Diskussion}

\subsection{Einzelspalt}
Sowohl die Werte für die Spaltbreite $b$, die mit Hilfe des Mikroskops bestimmt wurden als auch die, die bei bei der Beugung aufgenommen und durch eine Ausgleichsrechnung bestimmt werden, sind alle ähnlich zu den Angaben des Herstellers. Alle Messreihen können durch den Zusammenhang aus Gleichung \ref{eqn:einzel} beschrieben werden. Die größte Abweichung liegt  beim kleinsten Spalt mit $b_1=0.075 \si{\milli\meter}$ vor. Die Werte der Abweichungen der beiden Methoden sind meist ähnlich. Insgesamt zeigen die mit dem Mikroskop bestimmten Spaltbreiten geringere Abweichungen.

\subsection{Doppelspalt}
Wie Abbildung \ref{fig:doppel} zeigt, gilt Gleichung \ref{eqn:doppel} für die Intensitätsverteilung eines Doppelspalts. Im Vergleich mit der eines Einzelspaltes (siehe Abbildung \ref{fig:einzel1} -  \ref{fig:einzel3}) besitzt diese näher beieinander liegende Nebenmaxima mit höherer Intensität. Der Abstand $s$ zwischen den Spalten kann sowohl mit dem Mikroskop als auch über die Beugung genauer bestimmt werden als die Spaltbreite $b$. Wie zuvor beim Einzelspalt liefert das Ausmessen mit dem Mikroskop genauere Ergebnisse.
