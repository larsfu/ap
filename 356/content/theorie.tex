\section{Ziel}
Es sollen das Schwingungsverhalten einer elektrischen Kette von Kapazitäten und Induktivitäten (siehe Abb. \ref{fig:tiefpass}) untersucht werden. Dazu gehören Durchlass- und Dispersionskurve und stehende Wellen.

\fig{content/bilder/1}{$LC$-Kette mit parallelgeschalteten Kondensatoren und in Reihe geschalteten Spulen \cite{anleitung356}.}{tiefpass}

\section{Theorie}
\label{sec:Theorie}

\subsection{Schwingungsgleichung im stationären Fall}

Anhand der Kirchhoffschen Gesetze (hier Knotenregel) lässt sich für die unendliche $LC$-Kette (siehe Abb.~\ref{fig:kette}) eine Schwingugsgleichung mit
\begin{align}
  I_n - I_{n+1} + I_{n_\text{quer}} &= 0 \\
  \shortintertext{zu}
  \label{eqn:eins}
  -\omega^2 C U_n + \frac{1}{L} (-U_{n-1} + 2U_n - U_{n+1}) &= 0
\end{align}
\begin{align}
  \intertext{herleiten, wobei diese mit dem Ansatz}
  U_n(t) &= U_0 \symup{e}^{i \omega t} \symup{e}^{-i n \theta t}
\end{align}
\begin{align}
  \intertext{zur Lösung}
  -\omega^2 LC + 2 - (\symup{e}^{i} + \symup{e}^{-i\theta}) &= 0
\end{align}
führt. Diese wiederum ergibt die Dispersionsrelation
\begin{equation}
  \omega^2 = \frac{2}{LC}(1-\cos \theta)
\end{equation}
anhand derer ersichtlich ist, dass ein Schwingungsverhalten der Kette nur bei Frequenzen
\begin{equation}
  0 \le \omega < \frac{\sqrt{2}}{\sqrt{LC}}
\end{equation}
möglich ist.

Ähnlich verhält es sich auch im allgemeineren Fall einer alternierenden Kette aus $L, C_1$ und $C_2$ (siehe Abb.~\ref{fig:kette2}), wobei (\ref{eqn:eins}) dann zu
\begin{align}
  -\omega^2 C_1 U_{2n+1} + \frac{1}{L} (-U_{2n} + 2 U_{2n+1} - U_{2n+2}) &= 0
  \shortintertext{und}
  -\omega^2 C_2 U_{2n} + \frac{1}{L} (-U_{2n-1} + 2 U_{2n+1} - U_{2n+1}) &= 0
\end{align}
verallgemeinert wird. Durch die Lösung der Gleichungen mit einem ähnlichen Ansatz wie zuvor ergibt sich für die alternierende Kette eine Dispersionsrelation der Form
\begin{equation}
  \label{eqn:disprel}
  \omega_{1,2}^2 = \frac{1}{L} \left(\frac{1}{C_1} + \frac{1}{C_2}\right) \pm \frac{1}{L} \sqrt{\left(\frac{1}{C_1} + \frac{1}{C_2}\right)² - \frac{4\sin^2(\theta)}{C_1C_2}}.
\end{equation}
Demnach entstehen bei dieser Kette zwei Zweige der Dispersionskurve, bedingt durch die nötige Fallunterscheidung am $\pm$. Jene sind in Abb.~\ref{fig:alter} grafisch dargestellt.

\fig{content/bilder/4}{Ströme und Spannungen in einer $LC$-Kette, aus \cite{anleitung356}.}{kette}
\fig{content/bilder/5}{$LC$-Kette mit alternierenden Kondensatoren, aus \cite{anleitung356}.}{kette2}
\fig{content/bilder/7}{Dispersionskurven der alternierenden Kette, aus \cite{anleitung356}.}{alter}

\subsection{Mechanisches Analogon}
Die $LC$-Kette weist starke Ähnlichkeiten zu einem mechanischen System gekoppelter Oszillatoren (siehe Abb.~\ref{fig:federn}) auf. Die Schwingungsgleichung
\begin{equation}
  -\omega^2 M u_n + \eta (-u_{n-1} + 2 u_n + u_{n+1}) = 0
\end{equation}
mit der Federkonstanten $\eta$, Masse $M$ und Elongation $u_n(t)$ ähnelt (\ref{eqn:eins}). Auch hierbei kann der Oszillator durch Erweiterung zu einer alternierenden Kette mit $M \neq m$ verallgemeinert werden. Wieder ergibt sich eine Dispersionsrelation
\begin{equation}
  \omega_{1,2}² = \eta\left(\frac{1}{m} + \frac{1}{M}\right) \pm \eta \sqrt{\left(\frac{1}{M} + \frac{1}{m}\right)^2 - \frac{4\sin^2(\theta)}{mM}}
\end{equation}
die genau wie (\ref{eqn:disprel}) zwei Lösungszweige enthält. In einem Kristall können elektromagnetische Wellen Schwingungen im oberen der Zweige auslösen, daher wird dieser auch optischer Zweig genannt. Der untere Zweig hingegen akustisch, da er \enquote{normale} Schwingungen der Gitteratome beschreibt.
\fig{content/bilder/6}{Mechanisches Analogon zur $LC$-Kette, aus \cite{anleitung356}.}{federn}

\subsection{Phasengeschwindigkeit}
In der $LC$-Kette breiten sich Wellen mit der Wellengleichung
\begin{equation}
  U(t, n) = U_0 \symup{e}^{i(\omega t -n\theta)}
\end{equation}
aus, wobei $\theta$ die Phasenverschiebung pro Kettenglied darstellt. Die Phasengeschwindigkeit der Welle ist definiert als
\begin{align}
  v_\text{ph} &= \frac{\Delta n}{\Delta t}
  \intertext{woraus über die Phase}
  \Phi &= \omega t - n \theta \\
  v_\text{ph} &= \frac{\omega}{\theta}
\end{align}
folgt.

Mithilfe der Dispersionsrelation ergibt sich daraus dann
\begin{equation}
  \label{eqn:vph}
  v_\text{ph} = \frac{\omega}{\arccos\left(1-\frac{1}{2} \omega^2 LC\right)}.
\end{equation}

\subsection{Impedanz}
Von Interesse ist weiterhin die Eingangsimpedanz $Z$ der $LC$-Kette. Diese kann erneut unter Zuhilfenahme der Kirchhoffschen Gesetze (Siehe Abb.~\ref{fig:imp}) ermittelt werden. Aus der Wellengleichung folgt letztendlich
\begin{equation}
  \label{eqn:wellnw}
  \mathfrak{Z} = \frac{\mathfrak{u}_0}{\mathfrak{i}_0} = \frac{\omega L}{\sin \theta} = \frac{\sqrt{\frac{L}{C}}}{\sqrt{1-\frac{1}{4} \omega² LC}}.
\end{equation}

Daraus folgt, dass die Impedanz einer \emph{unendlich} langen Kette keinen Imaginärteil hat, also im Endeffekt einen Ohmschen Widerstand darstellt. Um dies nun mit einer \emph{endlichen} Kette nachzubilden wird ein Abschlusswiderstand an die Kette angeschlossen (Abb~\ref{fig:awid}), welcher Reflexionen am Ende der Kette, die zu imaginären Impedanzen führen könnten, vernichtet. Da der Wellenwiderstand der Kette nach (\ref{eqn:wellnw}) frequenzabhängig ist (siehe Abb.~\ref{fig:freqab}), muss für eine vollständige Verhinderung von Reflexionen auch der Abschlusswiderstand variabel angepasst sein.

\fig{content/bilder/12}{Schaltbild zur Bestimmung der Impedanz der $LC$-Kette, aus \cite{anleitung356}.}{imp}
\fig{content/bilder/14}{Unedliche $LC$-Kette mit Abschlusswiderstand, aus \cite{anleitung356}.}{awid}
\fig{content/bilder/15}{Frequenzabhängigkeit der charakteristischen Impedanz einer $LC$-Kette, aus \cite{anleitung356}.}{freqab}


\subsection{Stehende Wellen}
Im allgemeineren Fall, dass der Abschlusswiderstand nicht genau dem Wellenwiderstand der endlichen Kette entspricht (siehe Abb.~\ref{fig:wellenwiderstand}), können aufgrund der Reflexion am Kettenende bei bestimmten Frequenzen stehende Wellen auf der Kette beobachtet werden. Im Einzelnen sind verschiedene Fälle zu untersuchen:
\fig{content/bilder/20}{Abschluss einer endlich langen $LC$-Kette mit einem (komplexen) Widerstand $Z$, aus \cite{anleitung356}.}{wellenwiderstand}

$Z = \infty${}: Offenes Ende
Die Welle wird ohne Phasensprung vollständig reflektiert, es ist
\begin{equation}
  u_\text{ref} = u_E
\end{equation}
mit der hinlaufenden Spannung $u_E$ und reflektierten Spannung $u_\text{ref}$.

$Z = 0${}: Geschlossenes Ende
Am Ende wird die Welle mit Phasenverschiebung $\symup{\pi}$ vollständig reflektiert, also ist
\begin{equation}
  u_\text{ref} = -u_E.
\end{equation}

$Z = Z_\text{Kette}${}: Abschluss mit Wellenwiderstand
\begin{equation}
  u_\text{ref} = 0,
\end{equation}
also wird nichts reflektiert, es kann nicht zu einer stehenden Welle kommen.

Bedingung für das Auftreten einer stehenden Welle
Im offenen Fall lässt sich aus den Schwingungsgleichungen die Bedingung
\begin{equation}
  n_\text{max} \theta_k = k \symup{\pi}
\end{equation}
mit $k = 1,2,…,n_\text{max}$ für eine stehende Welle ermitteln, im geschlossenen Fall lautet diese
\begin{equation}
  n_\text{max} \theta_l = l \frac{\symup{\pi}}{2}
\end{equation}
mit $l = 0,1,3,5,…,2 n_\text{max} -1$.

Wenn diese Bedingungen erfüllt werden ergeben sich stehende Wellen, wie beispielhaft in Abb.~\ref{fig:stewe} dargestellt
\fig{content/bilder/17}{Beispiele von stehenden Wellen, aus \cite{anleitung356}.}{stewe}
