\section{Durchführung}
\label{sec:Durchführung}

Zum Ablesen der Brückenspannungen dient ein Oszilloskop.

\subsection{Wheatstonesche Brücke}
Die Wheatstonesche Brücke wird nach Schlatung \ref{fig:wheatstone} aufgebaut. Das Potetiometer muss so eingestellt werden,dass die Brückenspannung Null wird. Die Werte für die verschiedenen Widerstände werden notiert. Es werden zwei unbekannte Widerstände mit je drei verschiedenen Werte für $R_{2}$ ausgemessen.

\subsection{Kapazitätsmessbrücke}
Die Kapazitätsmessbrücke wird entsprechend \ref{fig:kapazität} aufgebaut. Es sollen zwei Kapazitäten der jeweiligen Kondensatoren gemsesen werden.
Der Widerstand $R_{2}$ wird hierbei nicht benötigt, da die Kondensatoren ausreichend kleine Innenwiderstände besitzen. Erneut wird das Potentiometer so eingestellt, dass die Brückenspannung gegen Null geht. Die anderen Werte werden aufgenommen.
Beim Ausmessen der RC-Kombination wird $R_{2}$ eingebaut. Anschließend müssen $R_{2}$ und das Potentiometer abwechselnd so variiert werden, dass die Brückenspannung null wird.

\subsection{Induktivitätsmessbrücke}
 Zum Aufbau der Induktivitätsmessbrücke dient \ref{fig:induktivität}. Wie bei der Messung unter Verwendung der Kapazitätsmessbrücke werden $R_{2}$ und das Potentiometer abwechselnd so eingestellt, dass die Brückenspannung verschwindet.

\subsection{Maxwell-Brücke}
Die Maxwell-Brücke wird nach \ref{fig:maxwell}aufgebaut. Die Spule, welche bereits durch die Induktivitätsmessbrücke, ausgemessen wurde, wird nun mit der Maxwell-Brücke ausgemessen. Die Widerstände $R_3$ und $R_4$ werden variiert, bis die Brückenspannung verschwindet. Nach Austauschen von $R_{2}$ wird die Messung wiederholt.

\subsection{Wien-Robinson-Brücke}
Die Schaltung wird nach \ref{fig:wien-robinson} aufgebaut. Die Messung wird im Bereich von 20-30000 \si{\Hz} durchgeführt. Die Werte der Frequenz, der Brücken- und der Speisespannung werden aufgenommen.
