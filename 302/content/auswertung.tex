\section{Auswertung}
\label{sec:Auswertung}

\subsection{Fehlerrechnung}

\subsubsection{Mittelwert und Standardabweichung}
Der Mittelwert mehrerer Messwerte wird berechnet durch
\begin{equation}
\langle v\rangle = \frac{1}{N} \sum_{i=1}^N v_i,
\end{equation}
dabei ist die Standardabweichung
\begin{equation}
s_i = \sqrt{\frac{1}{N - 1} \sum_{j=1}^N \left(v_j - \langle v\rangle\right){^2}},
\end{equation}
wobei $v_j$ ($j = 1, ..., N$) die Messwerte sind.
Der Standardfehler ist über
\begin{equation}
\sigma_i = \frac{s_i}{\sqrt{N}} = \sqrt{\frac{\sum_{j=1}^N \left(v_j - \langle v_i\rangle\right){^2}}{N \left(N - 1 \right)}}.
\end{equation}
definiert.


\subsubsection{Gaußfehler}
Bei einer fehlerbehafteten Funktion $f$ mit $k$ als fehlerbehafteter Größe und $\sigma_k$ als Ungenauigkeit, gilt
\begin{equation}
\Delta x_k = \frac{\mathrm{d}f}{\mathrm{d}k}\sigma_k.
\end{equation}

Der relative Gaußfehler berechnet sich nach
\begin{equation}
\Delta x_\text{k, rel} = 1 \pm \frac{\Delta x_k}{|x|}\cdot 100\%.
\end{equation}

Der absolute Gaußfehler ergibt sich aus
\begin{equation}
\Delta x_i = \sqrt{\left(\frac{\mathrm{d}f}{\mathrm{d}k_{1}}\cdot \sigma_{k_{1}}\right)^2 + \left(\frac{\mathrm{d}f}{\mathrm{d}k_{2}}\cdot \sigma_{k_{2}}\right)^2 + ...}.
\end{equation}

\subsubsection{Lineare Regression}
\label{sec:linregress}
Bei einer linearen Regression über den Messdaten ${x_i, y_i}$ wird für die Steigung
\begin{align}
  m &= \frac{\langle x y \rangle - \langle x \rangle \langle y \rangle}{\langle x^2 \rangle - \langle x \rangle ^2}
  \intertext{und für den $y$-Achsenabschnitt}
  b &= \langle y \rangle - m \cdot \langle x\rangle
\end{align}
angenommen. Für die Standardabweichung gelten
\begin{align}
  s_m &= \sqrt{\frac{1}{N-2} \sum^N_{i=1} (y_i - b - mx_i)^2}
  \shortintertext{und}
  s_b &= s_m \cdot \sqrt{\frac{1}{N (\langle x^2 \rangle - \langle x \rangle ^2)}}.
\end{align}

%\subsubsection{Lineare Regression}
%\begin{equation}
%\sigma {^2} = \sum_{k=1}^{N} \left(y_k - \left(\frac{\overline{xy} - \overline{x}\cdot\overline{y}}{\overline{x^2} - (\overline{x}){^2}}x_k + \left(\overline{y} - %\overline{B}\overline{x}\overline{B}\right)\right)\right){^2}
%\end{equation}

\subsection{Wheatstonesche Brücke}
Die Widerstandswerte 10 und 12 wurden mit jeweils 3 verschiedenen Referenzwiderständern $R_2$ nach (\ref{eqn:wheatstone}) vermessen. Die Ergebnisse finden sich in Tabelle \ref{tab:a}.

\begin{table}
  \centering
  \caption{Ergebnisse der Widerstandsmessbrücke.}
  \label{tab:a}
  \sisetup{
    round-mode=figures,
    round-precision=3
  }

  \begin{tabular}{
    l@{}
    S[table-format=4.0] @{${}\pm{}$} S[table-format=1.4]
    S[table-format=1.3] @{${}\pm{}$} S[table-format=1.6]
    S[table-format=3.0] @{${}\pm{}$} S[table-format=1.3]
    S[table-format=1.3] @{${}\pm{}$} S[table-format=1.6]
    S[table-format=3.0] @{${}\pm{}$} S[table-format=1.3]}
    \toprule
    \multicolumn{3}{c}{} & \multicolumn{4}{c}{Wert 10} & \multicolumn{4}{c}{Wert 12} \\
    %\midrule
    &\multicolumn{2}{c}{$R_{2} / \Omega$} &
    \multicolumn{2}{c}{$\sfrac{R_3}{R_4}$} &
    \multicolumn{2}{c}{$R_{10} / \Omega$} &
    \multicolumn{2}{c}{$\sfrac{R_3}{R_4}$} &
    \multicolumn{2}{c}{$R_{12} / \Omega$} \\
    \midrule
    \input{build/a1.txt}
    \midrule
    \multicolumn{5}{r}{Mittelwert}\input{build/a2.txt}
    \bottomrule
  \end{tabular}
\end{table}

\subsection{Kapazitätsmessbrücke}
Die Kapazitätswerte 1, 3 und 9 werden mit jeweils 3 verschiendenen Referenzkondensatoren $C_2$ nach (\ref{eqn:kapazität1}) und (\ref{eqn:kapazität2}) vermessen. Die Ergebnisse finden sich in Tabelle \ref{tab:b1} und Tabelle \ref{tab:b2}.

\begin{table}
  \centering
  \caption{Ergebnisse der Kapazitätsmessbrücke.}
  \label{tab:b1}
  \sisetup{
    zero-decimal-to-integer=true,
    round-mode=figures,
    round-precision=3
  }

  \begin{tabular}{
    l@{}
    S[table-format=3.0] @{${}\pm{}$} S[table-format=1.4]
    S[table-format=3.0] @{${}\pm{}$} S[table-format=2.2]
    S[table-format=1.3] @{${}\pm{}$} S[table-format=1.6]
    S[table-format=3.0] @{${}\pm{}$} S[table-format=2.1]
    S[table-format=3.0] @{${}\pm{}$} S[table-format=1.3]
    }
    \toprule
    %\multicolumn{3}{c}{} & \multicolumn{4}{c}{Wert 10} & \multicolumn{4}{c}{Wert 12} \\
    %\midrule
    &\multicolumn{2}{c}{$C_{2} / \si{\nano\farad}$} &
    \multicolumn{2}{c}{$R_2 / \si{\ohm}$} &
    \multicolumn{2}{c}{$\sfrac{R_3}{R_4}$} &
    \multicolumn{2}{c}{$R_x / \si{\ohm}$} &
    \multicolumn{2}{c}{$C_x / \si{\nano\farad}$} \\
    \midrule
    \multicolumn{11}{c}{Wert 1}\\
    \input{build/b1.txt}
    \midrule
    \multicolumn{11}{c}{Wert 3}\\
    \input{build/b2.txt}
    \midrule
    \multicolumn{11}{c}{Wert 9}\\
    \input{build/b3.txt}
    %\multicolumn{5}{r}{Mittelwert}\input{build/a2.txt}
    \bottomrule
  \end{tabular}
\end{table}
\begin{table}
  \centering
  \caption{Gemittelte Ergebnisse der Kapazitätsmessbrücke.}
  \label{tab:b2}
  \sisetup{
    zero-decimal-to-integer=true,
    round-mode=figures,
    round-precision=3
  }

  \begin{tabular}{
    l@{}
    c
    S[table-format=3.0] @{${}\pm{}$} S[table-format=1.2]
    S[table-format=3.0] @{${}\pm{}$} S[table-format=1.2]
    }
    \toprule
    &&
    \multicolumn{2}{c}{$R_x / \si{\ohm}$} &
    \multicolumn{2}{c}{$C_x / \si{\nano\farad}$} \\
    \midrule
    \input{build/b4.txt}
    \bottomrule
  \end{tabular}
\end{table}

\subsection{Induktivitätsmessbrücke}
Die Induktivität 16 wird mit zwei verschiendenen Referenzspulen $L_2$ nach (\ref{eqn:ind1}) und (\ref{eqn:ind2}) vermessen. Die Ergebnisse finden sich in Tabelle \ref{tab:c}.

\begin{table}
  \centering
  \caption{Ergebnisse der Induktivitätsmessbrücke.}
  \label{tab:c}
  \sisetup{
    round-mode=figures,
    round-precision=3
  }

  \begin{tabular}{
    l@{}
    S[table-format=2.1] @{${}\pm{}$} S[table-format=1.5]
    S[table-format=2.1] @{${}\pm{}$} S[table-format=1.2]
    S[table-format=1.2] @{${}\pm{}$} S[table-format=1.5]
    S[table-format=3.0] @{${}\pm{}$} S[table-format=2.1]
    S[table-format=3.0] @{${}\pm{}$} S[table-format=1.4]
    }
    \toprule
    &\multicolumn{2}{c}{$L_{2} / \si{\micro\henry}$} &
    \multicolumn{2}{c}{$R_2 / \si{\ohm}$} &
    \multicolumn{2}{c}{$\sfrac{R_3}{R_4}$} &
    \multicolumn{2}{c}{$R_{16} / \si{\ohm}$} &
    \multicolumn{2}{c}{$L_{16} / \si{\micro\henry}$} \\
    \midrule
    \input{build/c1.txt}
    \midrule
    \multicolumn{7}{r}{Mittelwert} \input{build/c2.txt}
    \bottomrule
  \end{tabular}
\end{table}

\subsection{Maxwellbrücke}
Zum Vergleich wird die Messung der Induktivität 16 mit der Maxwellbrücke wiederholt, Ergebnisse nach (\ref{eqn:max1}) und (\ref{eqn:max2}) finden sich in Tabelle \ref{tab:d}.

\begin{table}
  \centering
  \caption{Ergebnisse der Maxwellbrücke.}
  \label{tab:d}
  \sisetup{
    round-mode=figures,
    round-precision=3
  }

  \begin{tabular}{
    l@{}
    S[table-format=3.0] @{${}\pm{}$} S[table-format=1.4]
    S[table-format=4.0] @{${}\pm{}$} S[table-format=2.2]
    S[table-format=3.0] @{${}\pm{}$} S[table-format=2.2]
    S[table-format=3.0] @{${}\pm{}$} S[table-format=1.2]
    S[table-format=3.0] @{${}\pm{}$} S[table-format=2.1]
    S[table-format=3.0] @{${}\pm{}$} S[table-format=1.2]
    }
    \toprule
    &\multicolumn{2}{c}{$C_{4} / \si{\nano\farad}$} &
    \multicolumn{2}{c}{$R_2 / \si{\ohm}$} &
    \multicolumn{2}{c}{$R_3$} &
    \multicolumn{2}{c}{$R_4 / \si{\ohm}$} &
    \multicolumn{2}{c}{$R_{16} / \si{\ohm}$} &
    \multicolumn{2}{c}{$L_{16} / \si{\micro\henry}$} \\
    \midrule
    \input{build/d1.txt}
    \midrule
    \multicolumn{9}{r}{Mittelwert} \input{build/d2.txt}
    \bottomrule
  \end{tabular}
\end{table}

\subsection{Wien-Robinson-Brücke}
Die Frequenzabhängigkeit einer Wien-Robinson-Brücke wurde gemessen, Ergebnisse finden sich in Tabelle \ref{tab:e}, Abb. \ref{fig:e1} und Abb. \ref{fig:e2}. Die Theoriekurven in den Graphen ergeben sich aus (\ref{eqn:wr}), $\nu_0^\mathrm{ideal}$ aus (\ref{eqn:resonanz}).

\begin{table}
  \centering
  \caption{Resonanzfrequenz der Wien-Robinson-Brücke.}
  \label{tab:e}
  \sisetup{
    round-mode=figures,
    round-precision=3
  }

  \begin{tabular}{
    l@{}
    S[table-format=3.0] @{${}\pm{}$} S[table-format=1.3]
    S[table-format=3.0] @{${}\pm{}$} S[table-format=1.2]
    S[table-format=4.0]
    S[table-format=4.0] @{${}\pm{}$} S[table-format=1.2]
    S[table-format=1.3] @{${}\pm{}$} S[table-format=1.5]}
    \toprule
    &\multicolumn{2}{c}{$R / \si{\ohm}$} &
    \multicolumn{2}{c}{$C_3 / \si{\nano\farad}$} &
    {$\nu_0^\mathrm{real} / \si{\hertz}$} &
    \multicolumn{2}{c}{$\nu_0^\mathrm{ideal} / \si{\hertz}$} &
    \multicolumn{2}{c}{$\nu_0^\mathrm{real}/\nu_0^\mathrm{ideal}$} \\
    \midrule
    \begin{table}
        \caption{Messdaten, Hysteresekurve.}
        \centering
        \label{de}
        \begin{tabular}{l@{}cc|cc|cc} \toprule & {$I/\si{A}$}& {$B/\si{mT}$}& {$I/\si{A}$}& {$B/\si{mT}$}& {$I/\si{A}$}& {$B/\si{mT}$}\\\midrule& 0,0 & 27,05 & 3,5 & 792,5 & 3,0 & 730,1 \\
& 0,5 & 125,2 & 4,0 & 888,8 & 2,5 & 623,7 \\
& 1,0 & 242,6 & 4,5 & 978,0 & 2,0 & 508,0 \\
& 1,5 & 364,7 & 5,0 & 1059 & 1,5 & 394,5 \\
& 2,0 & 480,2 & 4,5 & 1000 & 1,0 & 274,3 \\
& 2,5 & 590,5 & 4,0 & 921,8 & 0,5 & 150,4 \\
& 3,0 & 692,7 & 3,5 & 829,6 & 0,0 & 27,67 \\
 \bottomrule \end{tabular} \end{table}

    \bottomrule
  \end{tabular}
\end{table}

\begin{figure}
  \centering
  \includegraphics{build/e1.pdf}
  \caption{Frequenzabhängigkeit der Wien-Robinson-Brücke.}
  \label{fig:e1}
\end{figure}

\begin{figure}
  \centering
  \includegraphics{build/e2.pdf}
  \caption{Vergrößerung von Abb. \ref{fig:e1} in der Nähe der Resonanzfrequenz.}
  \label{fig:e2}
\end{figure}

\subsection{Klirrfaktor}
Zuletzt wurde der Klirrfaktor $k$ des Funktionsgenerators Bestimmt. Dieser ist nach (\ref{eqn:klirr1}) bzw. (\ref{eqn:klirr2}) definiert. Dabei gilt
\begin{equation}
  U_2 = \frac{U_{Br}(\nu = \nu_0)}{f(2)},
\end{equation}
wobei
\begin{equation}
  f(\Omega) = \frac{{U}_{Br}}{{U}_S}(\Omega)
\end{equation}
aus (\ref{eqn:wr}) darstellt. Das Ergebnis findet sich in Tabelle \ref{tab:f}.

\begin{table}
  \centering
  \caption{Klirrfaktor.}
  \label{tab:f}
  \sisetup{
    round-mode=figures,
    round-precision=3
  }

  \begin{tabular}{
    l@{}
    S[table-format=1.2]
    S[table-format=1.3]
    S[table-format=1.5]}
    \toprule
    &
    {$U_{Br}(\nu = \nu_0) / \si{\milli\volt}$} &
    {$f(2)$} &
    {$k$} \\
    \midrule
    \input{build/f.txt}
    \bottomrule
  \end{tabular}
\end{table}

\newpage
\subsection{Messwerte}

\begin{table}
  \centering
  \caption{Messwerte der Wheatstone-Brücke.}
  \label{tab:a_mess}
  \sisetup{
    round-mode=off,
    round-precision=3
  }

  \begin{tabular}{
    l@{}
    S[table-format=4.1]
    S[table-format=3.1]
    S[table-format=3.1]
    S[table-format=3.1]
    S[table-format=3.1]}
    \toprule
    &&\multicolumn{2}{c}{Wert 10} & \multicolumn{2}{c}{Wert 12} \\
    &{$R_2 / \si{\ohm}$} &
    {$R_3 / \si{\ohm}$} &
    {$R_4 / \si{\ohm}$} &
    {$R_3 / \si{\ohm}$} &
    {$R_4 / \si{\ohm}$} \\
    \midrule
    \input{build/a_mess.txt}
    \bottomrule
  \end{tabular}
\end{table}

\begin{table}
  \centering
  \caption{Messwerte der Kapazitätsmessbrücke.}
  \label{tab:b_mess1}
  \sisetup{
    round-mode=off,
    round-precision=3
  }

  \begin{tabular}{
    l@{}
    S[table-format=3.1]
    S[table-format=3.1]
    S[table-format=3.1]
    S[table-format=3.1]
    S[table-format=3.1]
    S[table-format=3.1]
    S[table-format=3.1]
    S[table-format=3.1]}
    \toprule
    &\multicolumn{4}{c}{Wert 1} & \multicolumn{4}{c}{Wert 3} \\
    &{$C_2 / \si{\nano\farad}$} &
    {$R_2 / \si{\ohm}$} &
    {$R_3 / \si{\ohm}$} &
    {$R_4 / \si{\ohm}$} &
    {$C_2 / \si{\nano\farad}$} &
    {$R_2 / \si{\ohm}$} &
    {$R_3 / \si{\ohm}$} &
    {$R_4 / \si{\ohm}$} \\
    \midrule
    \input{build/b_mess1.txt}
    \bottomrule
  \end{tabular}
%\end{table}

%\begin{table}
%  \centering
%  \caption{Messwerte der Kapazitätsmessbrücke.}
%  \label{tab:b_mess2}
%  \sisetup{
%    round-mode=off,
%    round-precision=3
%  }

  \begin{tabular}{
    l@{}
    S[table-format=3.1]
    S[table-format=3.1]
    S[table-format=3.1]
    S[table-format=3.1]}
    %\toprule
    &\multicolumn{4}{c}{Wert 9} \\
    &{$C_2 / \si{\nano\farad}$} &
    {$R_2 / \si{\ohm}$} &
    {$R_3 / \si{\ohm}$} &
    {$R_4 / \si{\ohm}$} \\
    \midrule
    \input{build/b_mess2.txt}
    \bottomrule
  \end{tabular}
\end{table}

\begin{table}
  \centering
  \caption{Messwerte der Induktivitätsmessbrücke.}
  \label{tab:c_mess}
  \sisetup{
    round-mode=off,
    round-precision=3
  }

  \begin{tabular}{
    l@{}
    S[table-format=2.1]
    S[table-format=2.1]
    S[table-format=3.1]
    S[table-format=3.1]}
    \toprule
    & \multicolumn{4}{c}{Wert 16} \\
    &{$L_2 / \si{\milli\henry}$} &
    {$R_2 / \si{\ohm}$} &
    {$R_3/ \si{\ohm}$} &
    {$R_4 / \si{\ohm}$} \\
    \midrule
    \input{build/c_mess.txt}
    \bottomrule
  \end{tabular}
\end{table}

\begin{table}
  \centering
  \caption{Messwerte der Maxwellbrücke.}
  \label{tab:d_mess}
  \sisetup{
    round-mode=off,
    round-precision=3
  }

  \begin{tabular}{
    l@{}
    S[table-format=3.1]
    S[table-format=4.1]
    S[table-format=3.1]
    S[table-format=3.1]}
    \toprule
    & \multicolumn{4}{c}{Wert 16} \\
    &{$C_4 / \si{\nano\farad}$} &
    {$R_2 / \si{\ohm}$} &
    {$R_3/ \si{\ohm}$} &
    {$R_4 / \si{\ohm}$} \\
    \midrule
    \input{build/d_mess.txt}
    \bottomrule
  \end{tabular}
\end{table}

\begin{table}
  \centering
  \caption{Messwerte der Wien-Robinson-Brücke.}
  \label{tab:e_mess}
  \sisetup{
    round-mode=off,
    round-precision=3
  }

  \begin{tabular}{
    l@{}
    S[table-format=4.0]
    S[table-format=3.1]|
    S[table-format=5.0]
    S[table-format=3.1]}
    \toprule
    &{$\nu / \si{\hertz}$} &
    {$U_\mathrm{Br} / \si{\volt}$}
    &{$\nu / \si{\hertz}$} &
    {$U_\mathrm{Br} / \si{\volt}$}\\
    \midrule
    \input{build/e_mess.txt}
    \bottomrule
  \end{tabular}
\end{table}
