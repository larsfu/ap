\section{Diskussion}
\label{sec:Diskussion}

Über die Genauigkeit der einzelnen Messwerte kann nichts ausgesagt werden, da die Referenzwerte nicht bekannt sind. Als problematisch anzumerken ist, dass bei Kapazitäts- und Induktivitätsmessbrücke teilweise $0$ für $R_2$ bzw. $R_x$ gemessen bzw. berechnet wird, damit wird auch der Fehler $0$, da lediglich eine relative Genauigkeit der Referenzbauteile angegeben wurde. Es könnte die Messung negativ beeinflusst haben, dass eins der beiden Potentiometer schwergängig war. Wie Tabelle \ref{tab:e} zeigt, war die Bestimmung der Resonanzfrequenz ausreichend genau, denn die Abweichung vom Idealwert liegt bei $\SI{4.1}{\%}$. Wie in Abb. \ref{fig:e1} und \ref{fig:e2} ersichtlich, sind beim Frequenzganz der Wien-Robinson-Brücke teils erhebliche Abweichungen vom Theoriewert vorhanden, diese sind vermutlich dadurch zu erklären, dass die Bauteile für den Messbereich um $\nu_0$ ausgelegt sind und in anderen Frequenzbereichen sich Störeinflüsse ergeben, weiterhin ist die Speisespannung $U_S$ nicht konstant. Aus Tabelle \ref{tab:f} ist weiterhin ersichtlich, dass der Klirrfaktor des Frequenzgenerators sehr gering ist, jedoch ist wiederum kein Vergleich mit Referenzwerten möglich, da unbekannt.
